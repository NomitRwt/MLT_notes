% Options for packages loaded elsewhere
\PassOptionsToPackage{unicode}{hyperref}
\PassOptionsToPackage{hyphens}{url}
%
\documentclass[
]{article}
\usepackage{amsmath,amssymb}
\usepackage{lmodern}
\usepackage{iftex}
\ifPDFTeX
  \usepackage[T1]{fontenc}
  \usepackage[utf8]{inputenc}
  \usepackage{textcomp} % provide euro and other symbols
\else % if luatex or xetex
  \usepackage{unicode-math}
  \defaultfontfeatures{Scale=MatchLowercase}
  \defaultfontfeatures[\rmfamily]{Ligatures=TeX,Scale=1}
\fi
% Use upquote if available, for straight quotes in verbatim environments
\IfFileExists{upquote.sty}{\usepackage{upquote}}{}
\IfFileExists{microtype.sty}{% use microtype if available
  \usepackage[]{microtype}
  \UseMicrotypeSet[protrusion]{basicmath} % disable protrusion for tt fonts
}{}
\makeatletter
\@ifundefined{KOMAClassName}{% if non-KOMA class
  \IfFileExists{parskip.sty}{%
    \usepackage{parskip}
  }{% else
    \setlength{\parindent}{0pt}
    \setlength{\parskip}{6pt plus 2pt minus 1pt}}
}{% if KOMA class
  \KOMAoptions{parskip=half}}
\makeatother
\usepackage{xcolor}
\IfFileExists{xurl.sty}{\usepackage{xurl}}{} % add URL line breaks if available
\IfFileExists{bookmark.sty}{\usepackage{bookmark}}{\usepackage{hyperref}}
\hypersetup{
  hidelinks,
  pdfcreator={LaTeX via pandoc}}
\urlstyle{same} % disable monospaced font for URLs
\setlength{\emergencystretch}{3em} % prevent overfull lines
\providecommand{\tightlist}{%
  \setlength{\itemsep}{0pt}\setlength{\parskip}{0pt}}
\setcounter{secnumdepth}{-\maxdimen} % remove section numbering
\ifLuaTeX
  \usepackage{selnolig}  % disable illegal ligatures
\fi

\author{}
\date{}

\begin{document}

\title{MLT Week-5}
\author{Sherry Thomas \\ 21f3001449}

\maketitle
\tableofcontents

\begin{abstract}
The week commences with an exploration of Supervised Learning, specifically focusing on the topic of Regression. The aim is to provide a comprehensive understanding of the underlying mechanism of this popular machine learning technique, and its various applications. Additionally, this study delves into the variants of Regression, including kernel regression, and examines the probabilistic aspects of the technique.
\end{abstract}

\hypertarget{introduction-to-supervised-learning}{%
\section{Introduction to Supervised
Learning}\label{introduction-to-supervised-learning}}

Supervised learning is a machine learning technique in which the
algorithm is trained on labeled data, where the target variable or the
outcome is known. The goal of supervised learning is to learn a mapping
between the input data and the corresponding output variable.

Given a dataset \(\{x_1, \ldots, x_n\}\) where \(x_i \in \mathbb{R}^d\),
let \(\{y_1, \ldots, y_n\}\) be the labels, where \(y_i\) could belong
to the following:

\begin{itemize}
\tightlist
\item
  Regression: \(y_i \in \mathbb{R}\) Example: Rainfall Prediction
\item
  Binary Classification: \(y_i \in \{0, 1\}\) Example: Distinguishing
  between cats and dogs
\item
  Multi-class Classification: \(y_i \in \{0, 1, \ldots, K\}\) Example:
  Digit classification
\end{itemize}

\hypertarget{linear-regression}{%
\section{Linear Regression}\label{linear-regression}}

Linear regression is a supervised learning algorithm used to predict a
continuous output variable based on one or more input features, assuming
a linear relationship between the input and output variables. The goal
of linear regression is to find the line of best fit that minimizes the
sum of squared errors between the predicted and actual output values.

Given a dataset \(\{x_1, \ldots, x_n\}\) where \(x_i \in \mathbb{R}^d\),
let \(\{y_1, \ldots, y_n\}\) be the labels, where
\(y_i \in \mathbb{R}\).

The goal of linear regression is to find the mapping between input and
output variables. i.e. \[
h: \mathbb{R}^d \rightarrow \mathbb{R}
\] The error for the above mapping function is given by, \[
\text{error}(h) = \sum _{i=1} ^n (h(x_i) - y_i) ^2)
\] This error can be as small as zero, and this is achieved when
\(h(x_i)=y_i \hspace{0.5em} \forall i\). But this may not be the desired
output as it may represent nothing more than memorizing the data and its
outputs.

The memorization problem can be mitigated by introducing a structure to
the mapping. The simplest structure being of the linear kind, we shall
use it as the underlying structure for our data. Let
\(\mathcal{H}_{\text{linear}}\) represent the solution space for the
mapping in the linear space. \[
\mathcal{H}_{\text{linear}}=\Biggr \lbrace{h_w: \mathbb{R}^d \rightarrow \mathbb{R} \hspace{0.5em} s.t. \hspace{0.5em} h_w(x)=w^Tx \hspace{0.5em} \forall w \in \mathbb{R}^d } \Biggr \rbrace
\] Therefore, our goal is, \begin{align*}
\min _{h \in \mathcal{H}_{\text{linear}}} \sum _{i=1} ^n (h(x_i) - y_i) ^2 \\
\text{Equivalently} \\
\min _{w \in \mathbb{R}^d} \sum _{i=1} ^n (w^Tx_i - y_i) ^2
\end{align*} Optimizing the above is the entire point of the linear
regression algorithm.

\newpage
\hypertarget{optimizing-the-error-function}{%
\section{Optimizing the Error
Function}\label{optimizing-the-error-function}}

The minimization equation can be rewritten in the vectorized form as, \[
\min _{w \in \mathbb{R}^d} || X^Tw - y || ^2 _2
\] Let this be a function of \(w\) and as follows: \begin{align*}
f(w) &= \min _{w \in \mathbb{R}^d} || X^Tw - y || ^2 _2 \\
f(w) &= (X^Tw - y)^T(X^Tw - y) \\
\therefore\bigtriangledown f(w) &= 2(XX^T)w - 2(Xy) \\
\end{align*} Setting the above equation to zero, we get \begin{align*}
(XX^T)w^* &= Xy \\
\therefore w^* &= (XX^T)^+Xy
\end{align*} where \((XX^T)^+\) is the pseudo-inverse of \(XX^T\).

On further analysis of the above equation, we can say that \(X^Tw\) is
the projection of the labels onto the subspace spanned by the features.

\hypertarget{gradient-descent}{%
\section{Gradient Descent}\label{gradient-descent}}

The normal equation for linear regression, as seen before, needs the
calculation of \((XX^T)^+\) which can computationally be of \(O(d^3)\)
complexity.

As we know \(w^*\) is the solution of an unconstrained optimization
problem, we can solve it using gradient descent. It is given by,
\begin{align*}
w^{t+1} &= w^t - \eta^t \bigtriangledown f(w^t) \\
\therefore \hspace{0.5em} w^{t+1} &= w^t - \eta^t \left [ 2(XX^T)w - 2(Xy) \right ]
\end{align*} where \(\eta\) is a scalar used to control the step-size of
the descent and \(t\) is the current iteration.

Even in the above equation, the calculation of \(XX^T\) is needed which
is a computational expensive task. Is there a way to adapt this further?

\hypertarget{stochastic-gradient-descent}{%
\subsection{Stochastic Gradient
Descent}\label{stochastic-gradient-descent}}

Stochastic gradient descent (SGD) is an optimization algorithm used in
machine learning for finding the parameters that minimize the loss
function of a model. In contrast to traditional gradient descent, which
updates the model parameters based on the entire dataset, SGD updates
the parameters based on a randomly selected subset of the data, known as
a batch. This results in faster training times and makes SGD
particularly useful for large datasets.

For every step \(t\), rather than updating \(w\) using the entire
dataset, we use a small randomly selected (\(k\)) data points to update
\(w\). Therefore, the new gradient is
\(2(\tilde{X}\tilde{X}^Tw^t - \tilde{X}\tilde{y})\) where \(\tilde{X}\)
and \(\tilde{y}\) is the small sample randomly selected from the
dataset. This is manageable because
\(\tilde{X} \in \mathbb{R}^{d \times k}\) which is considerably smaller
that \(X\).

After T rounds, we use, \[
w ^T _{SGD} = \frac{1}{T}  \sum _{i=1} ^T w^i
\] This has a certain guarantee to have optimal convergence partly
because of the randomness involved in it.

\hypertarget{kernel-regression}{%
\section{Kernel Regression}\label{kernel-regression}}

What if the data points lie in a non-linear sub-space? Just like in the
case of clustering non-linear data, we will use kernel functions here
too.

Let \(w^* = X\alpha^*\) for some \(\alpha^* \in \mathbb{R}^d\).
\begin{align*}
X\alpha^*&=w^*\\
\therefore X\alpha^* &=(XX^T)^+Xy \\
(XX^T)X\alpha^* &=(XX^T)(XX^T)^+Xy \\
(XX^T)X\alpha^* &=Xy \\
X^T(XX^T)X\alpha^* &=X^TXy \\
(X^TX)^2\alpha^* &=X^TXy \\
K^2\alpha^* &=Ky \\
\therefore \alpha^* &=K^{-1}y
\end{align*} where \(K \in \mathbb{R}^{n \times n}\) and \(K\) can be
obtained using a kernel function like the Polynomial Kernel or RBF
Kernel.

How to predict using alpha and the kernel function? Let
\(X_{test} \in R^{d \times m}\) be the test dataset. We predict by,
\begin{align*}
w^*\phi(X_{test}) &=  \sum _{i=1} ^n \alpha_i^* k(x_i, x_{test_i})
\end{align*} where \(\alpha_i^*\) gives the importance of the \(i^{th}\)
datapoint towards \(w^*\) and \(k(x_i, x_{test_i})\) shows how similar
\(x_{test_i}\) is to \(x_i\).

\hypertarget{probabilistic-view-of-linear-regression}{%
\section{Probabilistic View of Linear
Regression}\label{probabilistic-view-of-linear-regression}}

Given a dataset \(\{x_1, \ldots, x_n\}\) where \(x_i \in \mathbb{R}^d\),
let \(\{y_1, \ldots, y_n\}\) be the labels, where
\(y_i \in \mathbb{R}\). The probability of \(y_i\) given \(x_i\) is
given by, \[
P(y|X) \sim w^TX + \epsilon
\] where \(w \in R^d\) is an unknown but fixed value and \(\epsilon\) is
the noise corresponding to distribution \(\mathcal{N}(0, \sigma^2)\).

Because of the probabilistic nature of this problem, this can be viewed
as an estimation problem and the best solution can be approached using
Maximum Likelihood. This is given by, \begin{align*}
\mathcal{L}\left(w; \hspace{0.5em} \begin{array}{cccc}
x_1, x_2, \ldots, x_n \\
y_1, y_2, \ldots, y_n\\
\end{array} \right )
&= \prod _{i=1} ^n \left [ \frac{1}{\sqrt{2\pi}\sigma}  e^{\frac{-(w^Tx_i - y_i)^2}{2\sigma^2}} \right ] \\
\log\mathcal{L}\left(w; \hspace{0.5em} \begin{array}{cccc}
x_1, x_2, \ldots, x_n \\
y_1, y_2, \ldots, y_n\\
\end{array} \right )
&= \sum _{i=1} ^n \frac{-(w^Tx_i - y_i)^2}{2\sigma^2} \frac{1}{\sqrt{2\pi}\sigma}
\end{align*} \[
\text{Taking gradients w.r.t. } w \text{ on both sides, we get}
\] \[
\hat{w}_{ML} = w^* = (XX^T)^+Xy
\] Therefore, Maximum Likelihood Estimator assuming Zero mean Gaussian
Noise is the same as linear regression with square error.

\hypertarget{credits}{%
\section{Credits}\label{credits}}

Professor Arun Rajkumar: The content as well as the notations are from
his slides and lecture.

\end{document}
